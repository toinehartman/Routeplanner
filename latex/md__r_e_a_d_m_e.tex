This is a routeplanner, whick gives the shortest route between two points on an 5 by 5 grid. \subsection*{Synopsis}

This routeplanner find a route between a starting checkpoint on the edge of a {\bfseries 5 by 5 grid} and one or more endpoints. It can go by the checkpoints in a given order or via the shortest route that visits them all. 

\subsection*{Code Example}

``` ➜ ./bin/routeplanner.out 1 5 \begin{DoxyVerb}(0, 4)   9:(1, 4)   8:(2, 4)   7:(3, 4)     (4, 4)
\end{DoxyVerb}
 10\+:(0, 3) (1, 3) (2, 3) (3, 3) 6\+:(4, 3) 11\+:(0, 2) (1, 2) (2, 2) (3, 2) 5\+:(4, 2) 12\+:(0, 1) (1, 1) (2, 1) (3, 1) 4\+:(4, 1) (0, 0) 1\+:(1, 0) 2\+:(2, 0) 3\+:(3, 0) (4, 0)

Start 1\+: (1, 0) End 5\+: (4, 2)

(1, 0) ▷ (2, 0) ▷ (3, 0) ▷ (4, 0) ▷ (4, 1) ▷ (4, 2) ```

\subsection*{Motivation}

This program wa written to guide an autonomous robot, as part of E\+P\+O2, a project of Electrical Engineering at the University of Technology Delft.

\subsection*{Installation}

``` cd $<$download\+\_\+folder$>$/\+Routeplanner\textbackslash{} Toine/ make ``{\ttfamily  And then}./bin/routeplanner.out 1 5`, this will give the shortest route from checkpoint 1 to checkpoint 5.

\subsection*{Contributors}

Projectgroep B2\+:


\begin{DoxyItemize}
\item Daniël Brouwer
\item Toine Hartman
\item Dennis de Jong
\item Sam de Jong
\item George Koolman
\item Wesley Umans
\item Luc van Wietmarschen
\end{DoxyItemize}

The method to find the shortest route is based on \href{http://en.wikipedia.org/wiki/Lee_algorithm}{\tt Lee's algorithm}. 